\documentclass[10pt,b5paper,twoside,openany]{book}
% openany - normally chapters start at right page. That leads to blank pages if the content does not allow the chapter
% to start at the right page. openany function allows chapters to start at any page
%======================================================================
%>>>>>>> PACKAGES
\usepackage[utf8]{inputenc} %utf8 option to match Editor encoding
\usepackage[T1]{fontenc}  %some font encoding package
\usepackage{lmodern} %package that defines the font itself, allows polish letters
\usepackage{graphicx} %allows adding graphics to the document
\usepackage{wrapfig} %allows to create objects surrounded by text
\usepackage{amsmath,amsfonts,amssymb} %more math symbols, like matrices etc.
\usepackage[rightcaption]{sidecap} %allows to add captions at right and left sides of the figure, needs to be packed
%into the figure environment
\usepackage{float} %allows to specify floating options in some environments
\usepackage[margin=0.5in]{geometry} %margins
\usepackage{hyperref} %hyperlinks
\usepackage{xcolor} %
\usepackage{fancyvrb} %adds new types of verbatims
\usepackage{tcolorbox}
\usepackage{titlesec} %allows titling customization

\usepackage[acronym,toc,section=section]{glossaries}

%
% acronym - allows create different table for acronyms
% toc,section=section - allows glossaries to be printed in table of contents

\tcbuselibrary{skins,breakable} %probably for propper working of these black environments that code from VS is written in
%==========================================================================
% >>>>>>> DEFs
\def\bs{\textbackslash}
\def\us{\textunderscore}
\def\source_space{\vspace{0.2em}}

% >>>>>>> DEFINECOLORS
% href colors
\definecolor{hrefurl}{RGB}{46,71,217}
\definecolor{hreflink}{RGB}{39,117,15}

% >>>>>>> NEWENVIRONMENTS
% this is a custom verbatim with black background
\newenvironment{BGVerbatim}
 {\VerbatimEnvironment
  \begin{tcolorbox}[
    breakable,
    colback=vsblack,
    spartan
    ]%
  \begin{Verbatim}}
 {\end{Verbatim}\end{tcolorbox}}
 
% >>>>>>> NEWCOMMANDS
\newcommand{\tc}[2]{\textcolor{#1}{#2}}
\newcommand{\ul}[1]{\underline{#1}}
\newcommand{\myline}[2]{\noindent\makebox[\linewidth]{\rule{#1cm}{#2pt}}}
\newcommand{\keyshortcut}[1]{\texttt{#1}}
%==========================================================================
% >>>>>>> SETTINGS
\hypersetup %edytuje właściwości hiperłączy np. kolor
{
colorlinks=true,
linkcolor=black,
urlcolor=hrefurl
}

% \pagestyle{empty} %sprawia ze strony nie są numerowane - jezeli jest na stronie \maketitle to style jest nadpisywany i na takiej stronie trzeba dodatkowo umiescic zaraz po \maketitle komende \thispagestyle{empty}

\parindent 0px %ustawia wartość wcięcia na początku akapitu/paragrafu na zero, co daje taki efekt, że nie ma wcięć

\titleformat{\part}{\bfseries\centering\Huge{\titlerule[1.5pt]\vspace{0.1em}\titlerule[1.5pt]\vspace{0.5em}}}{Part \thepart\ -}{10pt}{\Huge}[{\vspace{0.5em}\titlerule[1.5pt]\vspace{0.1em}\titlerule[1.5pt]}]
\titlespacing{\part}{0pt}{0pt}{20pt}

\titleformat{\chapter}{\bfseries\centering\Huge{\titlerule[1.5pt]}}{\thechapter .}{10pt}{\huge}[{\titlerule[1.5pt]}]
\titlespacing{\chapter}{0pt}{0pt}{20pt}

\titleformat{\section}{\bfseries\large}{\thesection}{0.5em}{}[\titlerule]
\titlespacing{\section}{0pt}{1ex}{10pt} %how to use "plus" operator

\makeglossaries %>>>>>>>>>>>>>>>>>>>>>>>>>>>>>>>>>>>GLOSSARIES

%glossary entries examples:

%\newglossaryentry{paradigm}
%{
   % name=paradigm,
    %description={a typical example or pattern of something; a pattern or model}
%}

%\newacronym{cpu}{CPU}{Central Processing Unit}


\newglossaryentry{miktex}
{
name=MiKTeX ,
description={a free and open-source distribution of the TeX/\LaTeX typesetting system for Microsoft Windows}
}

\newglossaryentry{bibtex}
{
name=BibTeX ,
description={a reference management software for formatting lists of references. The BibTeX tool is typically used together with the \LaTeX document preparation system}
}

\newglossaryentry{gls-ctan} %% glossary connected with below acronym
{
name=Comprehensive TeX Archive Network,
description={the authoritative place where TeX related material and software can be found for download}
}
\newacronym[see={[Glossary:]{gls-ctan}}]{ctan}{CTAN}{Comprehensive TeX Archive Network\glsadd{gls-ctan}}

%>>>>>>>>>>>>>>>>>>>>>>>>>>>>>>>>>>>>>>>>>>>>>>>GLOSSARIES-END

\date{\vspace{-3em}}
\title{\vspace{-2em}{\Huge \texttt{>> Computer Science <<}}\vspace{-0.5em}}
\author{{\LARGE Created and typeset by: Tomasz Zdeb}}
%adding some negative vspace allows to remove unnecessary blank space

\begin{document} %>>>>>>> PREAMBLE END
\maketitle
% \thispagestyle{empty} % \maketitle nadpisuje dzialanie \pagestyle{empty} dlatego musimy zaraz po tej komendzie skorzystać z ustawienia dla tej konkretnej strony

\tableofcontents
\newpage

\part{\LaTeX}

\chapter{Essential concepts}

\section{Glossaries} \label{section:glossaries}

\Glspl{glossary} are being defined as ``\textit{an alphabetical list, with meanings, of the words or phrases in a text that are difficult to understand}"~\cite{cambridge_dictionary_glossary_definition}. To automate \glspl{glossary} generation and maintenence, \LaTeX\ uses a package called: \textbf{\glspl{glossary}}. Like in most cases, it is sometimes desired to specify additional \glspl{parameter} to the package, e.g. \textit{acronym}, \textit{toc}, \textit{section=section}. Which will be explained later in this section.\\

Before processing any code two things have to be mentioned. One: \glspl{glossary} package require \textbf{\href{https://www.perl.org/}{Perl}} interpreter to be present at the machine. Two: the package requires custom compilation scheme in order to take effect. Unfortunately there is no preset built into \textbf{\href{https://www.xm1math.net/texmaker/}{TeXMaker}} environment for a compilation scheme that will proces glossaries. A custom command \gls{pipeline} has to be configured manually.

\begin{figure}[H]
\centering
\includegraphics[scale=0.6]{content/LaTeX/figures/user_command_glossaries_marked.png}
\caption{Location of \textbf{Edit User Commands} button in \textbf{\href{https://www.xm1math.net/texmaker/}{TeXMaker}} environment}
\end{figure}

In order to define the custom command \gls{pipeline}, go to: \textbf{User} -> \textbf{User Commands} -> \textbf{Edit User Commands} and type below code into \textbf{command} field:
%\vspace{-0.5em} --------------------------------------------------------------------
\begin{verbatim}
pdflatex -synctex=1 -interaction=nonstopmode %.tex | makeglossaries % | pdflatex 
-synctex=1 -interaction=nonstopmode %.tex | "C:/Program Files (x86)/Adobe/Acrobat
Reader DC/Reader/AcroRd32.exe" %.pdf
\end{verbatim}

\begin{figure}[H]
\centering
\includegraphics[scale=0.6]{content/LaTeX/figures/custom_command_marked.png}
\caption{Layout of \textbf{Edit User Commands} window in \textbf{\href{https://www.xm1math.net/texmaker/}{TeXMaker}} environment}
\end{figure}

Pipe symbol is used to chain commands. Words preceded by pause: ``-'' are \glspl{parameter} passed to commands, which are words without any additions, placed at the beginning of every part - right after pipe or at the very beginning. Disassembly of this chain of commands allows to differentiate four differents parts here:
\begin{enumerate}
\item Generate \textbf{.pdf} file from the \textbf{.tex} file
\item Generate \glspl{glossary}
\item Generate \textbf{.pdf} file from the \textbf{.tex} file
\item Display \textbf{.pdf}
\end{enumerate}

There are two types of entries that \gls{glossary} package provides by default:
\begin{itemize}
\item \glspl{glossary}
\item acronyms
\end{itemize}

\Glspl{glossary} are being defined in the preamble after \texttt{\bs makeglossaries} command that has to be put before any of the entries. Syntax of an \gls{glossary} and acronym entry is as shown below.

\begin{figure}[H]
\centering
\includegraphics[scale=1.0]{content/LaTeX/figures/glossary_definition.png}
\caption{Commands used to define \glspl{glossary} and acronyms}
\label{fig:glossary_definition}
\end{figure}

To print a list of \glspl{glossary}, \texttt{\bs printglossary} command is used. For an entry to be printed on the list, at least one reference to it in the text is needed. Otherwise even if defined in the preamble, the entry won't be printet on the list.

\begin{figure}[H]
\centering
\includegraphics[scale=1.0]{content/LaTeX/figures/printglossary.png}
\caption{Commands used to print \gls{glossary} and acronym list}
\label{fig:printglossary}
\end{figure}

By default a list containing both \glspl{glossary} and acronyms is printed. As shown later on figure \ref{fig:usepackage_glossaries} \gls{parameter} \texttt{acronym} can be provided while invoking the package use, that allows to print separate list for acronyms and \glspl{glossary}. To print the separate list for acronyms specify \texttt{type=\bs acronymtype} attribute in a separate \texttt{\bs printglossary} call as shown on figure \ref{fig:printglossary}, which also illustrate, how a custom title\footnote{Title displayed in the text} as well as token title\footnote{Title displayed in table of content} can be set. To allow \gls{glossary} table to be included in table of contents an additional parameters: \texttt{toc} and \texttt{section=section} are needed when invoking \textbf{\glspl{glossary}} package as shown on picture \ref{fig:usepackage_glossaries}.

\begin{figure}[H]
\centering
\includegraphics[scale=0.8]{content/LaTeX/figures/glossary_types.png}
\caption{Lists of acronyms and glossaries generated separately with use of \textbf{acronym} parameter for \textbf{\glspl{glossary}} package, that allows for separate list for acronyms to be printed}
\end{figure}

To reference a \gls{glossary} or acronym, several commands are used.

\begin{figure}[H]
\centering
\includegraphics[scale=1.0]{content/LaTeX/figures/reference_glossaries.png}
\caption{Commands used to reference \glspl{glossary} and acronyms}
\end{figure}

Each entry can be referenced in several ways. Plural and singular forms as well as upper case and lowercase versions are avaliable for \glspl{glossary}. Acronym references differ from each other by level of detail printed.

\begin{figure}[H]
\centering
\includegraphics[scale=1.0]{content/LaTeX/figures/glossary_calls.png}
\caption{Examples of \glspl{glossary} and acronyms references}
\end{figure}

\begin{figure}[H]
\centering
\includegraphics[scale=1.0]{content/LaTeX/figures/usepackage_glossaries.png}
\caption{Command used to invoke the use of \textbf{glossaries} package with additional parameters passed to it}
\label{fig:usepackage_glossaries}
\end{figure}

There are many more options regarding to \gls{glossary} package, i.e. commands used to customize display of the \gls{glossary}, enable sorting, create custom \glspl{glossary}, put alternative text in references to a \gls{glossary}, define custom spelling for plural form, etc. You can find them on \acrshort{ctan} \href{https://www.ctan.org/pkg/glossaries}{website}~\cite{ctan_glossaries}.

\section{Bibliography}

In order to automate the process, make it more flexible and easily maintainable, there is a bibliography processor shipped with \Gls{miktex}, called \textbf{\Gls{bibtex}}. In general \textbf{\Gls{bibtex}} is considerd to be standalone tool, but in real life practice, it has very few uses, beside being bibliography management system for \LaTeX distributions. Thus it's very convinient to include it in \Gls{miktex} package by default.

\Gls{bibtex} itself alows to create citations to specified bibliography. It also contains built in bibliography listing functionality with predefined styles, that can be overriden.

\begin{figure}[H]
\centering
\includegraphics[scale=0.8]{content/LaTeX/figures/biblio_outcome.png}
\caption{An example showing default style of bibliography listing with citations generated by \gls{bibtex}}
\label{fig:bibliography_example}
\end{figure}

Bibliography items are stored in files with \textbf{\gls{bib}} extension, which are organized as a key-value dictionary. There is a set of allowed key values, to be used in a bibliography entry. There are no order restrictions on which key should be specified first, although it is usualy good to folow some convenctions like specifying the title first.

\begin{figure}[H]
\centering
\includegraphics[scale=0.9]{content/LaTeX/figures/biblio_example.png}
\caption{Structure of a bibliography item entry in a \gls{bib} file}
\label{fig:biblio_example}
\end{figure}

\begin{figure}[H]
\centering
\includegraphics[scale=0.9]{content/LaTeX/figures/biblio_bib.png}
\caption{Example of website entries in a \textbf{.bib} file}
\label{fig:biblio_websites}
\end{figure}

By default only entries that have been cited are printed in the bibliography listing. This behavious can be changed by using \texttt{\bs nocite\{*\}} command. It accepts keys as parameters, to print certain entries. Using asterix results in printing all results.

To perform a citation use \texttt{\bs cite\{...\}} command, replacing dots with bibliography entry name. It is also recommended to use tilde between the content and the command to span the citation with the content.

\begin{figure}[H]
\centering
\includegraphics[scale=0.9]{content/LaTeX/figures/biblio_latex.png}
\caption{Example showing how to use citing in the document}
\label{fig:biblio_usage}
\end{figure}

Bibtex compilation pipeline consists of several compilation steps. By processing \textbf{.bib} files, \textbf{.bbl} files are being produced, which are the direct source of content for \textbf{pdflatex}.

\begin{figure}[H]
\centering
\includegraphics[scale=0.9]{content/LaTeX/figures/biblio_bbl.png}
\caption{Example of a \textbf{.bbl} file structure}
\label{fig:biblio_bbl_example}
\end{figure}

Some quick guides, as well as some converters can be found at \href{https://www.bibtex.com/g/bibtex-format/}{\textbf{BibTeX} website}~\cite{bibtex_format} and \href{https://www.overleaf.com/learn/latex/Bibliography_management_with_bibtex}{Overleaf guide}~\cite{bibliography_management}. Additional information about the functionalities themselves can be found at \href{https://tug.org/bibtex/}{TeX Users Group}~\cite{tex_user_group_bibtex}.\\

Advanced users may want to create a bibliography after each part or chapter. Detailed guide on how to achieve such effets can be found at \href{https://tex.stackexchange.com/questions/229846/different-bibliographies-for-each-chapter-with-shared-references}{\textbf{TeX Stack Exchange issue}}~\cite{stack_bibliography_at_each_chapter}.

\section{Listing environments}

\subsection{Enumerate}

\begin{figure}[H]
\centering
\includegraphics[scale=0.8]{content/LaTeX/figures/enumerate_outcome_example.png}
\caption{Example of an enumerate environment}
\label{fig:enumerate_outcome_example}
\end{figure}

Source code:

\begin{Verbatim}
\begin{enumerate}
  \item item one
  \item item two
  \item item three
\end{enumerate} 
\end{Verbatim}

\subsection{Itemize}

\begin{figure}[H]
\centering
\includegraphics[scale=0.8]{content/LaTeX/figures/itemize_outcome_example.png}
\caption{Example of an itemize environment}
\label{fig:itemize_outcome_example}
\end{figure}

Source code:

\begin{Verbatim}
\begin{itemize}
  \item item one
  \item item two
  \item item three
\end{itemize} 
\end{Verbatim}

\subsection{Description}

Description environment can be used as a simple \hyperref[section:glossaries]{glossary}. For an advanced glossary management, it is recommended to use dedicated \hyperref[section:glossaries]{package}

\begin{figure}[H]
\centering
\includegraphics[scale=0.8]{content/LaTeX/figures/description_outcome_example.png}
\caption{Example of an description environment}
\label{fig:description_outcome_example}
\end{figure}

Source code:

\begin{Verbatim}
\begin{description}
  \item[Item one] description of item one
  \item[Item two] description of item two
  \item[Item three] description of item three
\end{description} 
\end{Verbatim}

\section{Float environments}

Float environment are being defined as containers for content that cannot be broken over a page~\cite{wiki_floats_figures_captions}. Their practical usage focuses on spanning a caption and label\footnote{Invisible identifier used for references throughout the document} as a single entity.\\

For example, there are dedicated environments, mostly for tables and figures where the table or figure itself can be put as well as it's label and caption.

\begin{Verbatim}
\begin{figure}[H]
  \centering
  \includegraphics[scale=0.8]{content/LaTeX/figures/description_outcome_example.png}
  \caption{Example of an description environment}
  \label{fig:description_outcome_example}
\end{figure}
\end{Verbatim}

Above source code has been used to create the figure from previous section. There are all three elements mentioned previously, as well as \texttt{\bs centering} command for additional control. Take note that the environment itself has an intuitive name that suggests it's usage.\\

Floating environment are not displayed as regular text. \LaTeX\ treats them as some kind of entity that can be moved over a page to provide the best content fit. As You can see at the example, there is \textbf{H} provided as additional argument to the environment opening tag. It's a way of forcing \LaTeX\ to display the environment exactly where it has been placed in the source code. To find out more about other positioning parameters, feel free to check the \href{https://en.wikibooks.org/wiki/LaTeX/Floats,_Figures_and_Captions}{Wikipedia article} regarding that matter.

\section{How to avoid a line break}
To instruct \LaTeX\ that two elements, for example like: two words or word and citation to be displayed together at any circumstances use tilde instead of whitespace. Tilde is a special character, that instructs the interpretted to do so.
\begin{Verbatim}
Some citation~\cite{bibliography_entry_name}
\end{Verbatim}

On the other hand, since tilde is a special character, there must be some way to display it as a character. There is - to do that, use \texttt{\bs textasciitilde} command.
\section{How to generate tilde}
\fbox{\textcolor{red}{FINISH THIS CHAPTER}}
\section{Differences between ref and hred referencing}
\fbox{\textcolor{red}{FINISH THIS CHAPTER}}
\section{MikTeX standard files}
\fbox{\textcolor{red}{FINISH THIS CHAPTER}}
\begin{verbatim}
C:\Program Files\MiKTeX\tex\latex\base
\end{verbatim}
\section{TO DOs}
\fbox{\textcolor{red}{Draw a diagram of this project}}\\
\fbox{\textcolor{red}{Prepare a guide on how to setup VSCode and create compilation scripts}}\\
\fbox{\textcolor{red}{Add list of tables and list of figures to the table}}\\
\fbox{\textcolor{red}{Add bibliography to the list of contents}}\\
\fbox{\textcolor{red}{Add non numbered chapter: acronyms and glossaries to table of contents}}\\
\fbox{\textcolor{red}{Go through the whole document and add lacking glossary/acronym entries}}\\
\fbox{\textcolor{red}{Go through the whole document and add lacking bibliography entries}}\\
\fbox{\url{http://www.peteryu.ca/tutorials/publishing/latex_captions}}\\
\fbox{\url{https://tex.stackexchange.com/questions/229846/different-bibliographies-for-each-chapter-with-shared-references}}\\
\fbox{\url{https://ctan.org/pkg/multirow}}\\
\fbox{\url{https://dictionary.cambridge.org/pl/dictionary/english/glossary}}\\
\fbox{\url{https://tex.stackexchange.com/questions/17653/how-to-list-all-bibliography-entries-without-citing}}\\
\fbox{\url{https://www.overleaf.com/learn/latex/Algorithms}}\\
\fbox{\url{https://tex.stackexchange.com/questions/1669/resuming-a-list}}\\

 %LaTeX Part

\part{Windows OS}

\section{Key Shortcuts}
\begin{itemize}
\item \keyshortcut{<Ctrl + R>} - starts \textbf{run window}
\item \keyshortcut{<Win + Shift + S>} runs \textbf{Snip \& Sketch} (screenshot)
\fbox{WIN + . - emoji menu}
\end{itemize}

\section{Run Commands}
\begin{itemize}
\item \keyshortcut{Run -> taskmgr} - runs \textbf{Task Manager}
\item \keyshortcut{Run -> winver} - runs simple program displaying \textbf{Windows} version on current machine
\item \keyshortcut{Run -> mspaint} - runs \textbf{Paint}
\item \keyshortcut{Run -> calc} - runs \textbf{Calculator}
\item \keyshortcut{Run -> control} - runs \textbf{Control Panel}
\end{itemize}

\section{Windows register}

It's a system database that contains most of systems configuration options. User or system settings as well as applications setting are stored in the register. Before \textbf{Windows 95} \texttt{.ini} configurations files were used (similarly to linux), but later all of the configuration data is stored in register. To modify the register a dedicated editor is used, passing \texttt{regedit} to \textbf{Windows Run} will launch it.\
 %Windows Part

\input{content/linux/linux} %Linux Part

\part{General Concepts}

\chapter{Learning Strategies}

\section{Top-down and bottom-up}

There are two main approaches when it comes to learning how to code:
\begin{itemize}
\item \textbf{top-down} - focuses on following a tutorial on how to make a full project without going too much into details 
\item \textbf{bottom-up} - focuses on learning basic concepts and all the details, then aggregating them into a bigger project
\end{itemize}
None of these approaches is perfect, both have their pros and cons, but to achieve quite a good learning effitiency it's optimal to combine these two in a learning process.\\

Building a working application gives a serious \textbf{load of satisfaction} that pushes the one deeper into learning process, that is the reason why learning should focus on \textbf{solving some real life problems}. It's good to learn programming in the incremental way - that means that it's necessary to always maintain working application, and develop only one (or even a part of) new feature at the time (to avoid scenarios when working on couple different features, and none of them is working as well as the whole application). This approach could be used not only in term of creating a single project but in term of whole learning process. Sometimes it's good to learn a little bit of one thing, then another, and then another, when gathered some general knowledge on these topics, come back to the first and master it, then to the second and then to the third.

\chapter{Content Versioning}

\section{Copying}

\section{Diff}

\section{GIT}

\section{SVN}

\chapter{What is...?}

\section{REST API}


Acronyms stand for \textbf{REpresentational State Transfer} and \textbf{Application Programming Interface}.\\

What's an API

The purpose of creating an \textbf{API} is to allow \textbf{application} or \textbf{service} access to a resource in other \textbf{application} or \textbf{service}. \textbf{Application} or \textbf{service }that contains the resource is then called \textbf{server}, and the \textbf{application} or \textbf{service} that accesses the resource is called \textbf{client}.\\

Stateles communication

\textbf{WEB APIs} don't have necessarily to be \textbf{RESTful}, the most characteristic feater of \textbf{REST} is that the communication is \textbf{stateless} (there is no session created, all the necessary information is contained in the passed request, and no information about previous requests is stored).\\

How todays REST APIs work

Most of todays \textbf{WEB Services} use \textbf{RESTful APIs} due to their flexibility and versatility. \textbf{REST} comparing to first generation \textbf{XML-RPC} protocol and second generation \textbf{SOAP} (that force to use a very specific structure of communication)	can be created by any programming language and use many different data formats, JSON is the most popular one though due to it's human readable form and simplicity.\\

\textbf{REST APIs} use \textbf{HTTP} requests to perform \textbf{CRUD} (\textbf{Create Read Update Delete}) opeartions on the resource. For example an \textbf{REST API} can use \texttt{GET} request to obtain a record, \texttt{POST} request to create a record, \texttt{PUT} request to update a record and \texttt{DELETE} request to delete a record.

Endpoints

To allow clients to acces the methods, a \textbf{path} is assigned to every method - this path is called the \textbf{endpoint}. Developers assign (map) these paths to a given method in \textbf{application} code.

\section{SOAP}

\chapter{Project Ideas} %General Concepts Part

\part{Final} 

\chapter*{Acronyms and Glossary}

\printglossary[title=Acronyms, toctitle=Acronyms,type=\acronymtype]

\printglossary[title=Glossary, toctitle=Glossary] 

\nocite{*} %allows to print all the bibliography entries defined in bib style even if they were not referenced. If instead of asterix a key would be present, then only the entry specified by that key would be printed.
\bibliographystyle{plain}
\bibliography{bibliography/sources}

\end{document}