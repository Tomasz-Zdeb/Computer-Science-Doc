% POCZĄTEK PREAMBUŁY
\documentclass[10pt,b5paper,twoside,openany]{book}
% openany - normally chapters start at right page. That leads to blank pages if the content does not allow the chapter
% to start at the right page. openany function allows chapters to start at any page
%======================================================================
%----------------------------------------------------------------------
\usepackage[utf8]{inputenc} % `utf8` option to match Editor encoding
\usepackage[T1]{fontenc}
\usepackage{lmodern}
%======================================================================
% DOŁĄCZONE PAKIETY
\usepackage{graphicx} %pozwala na dodawanie obrazów
\usepackage{wrapfig} %pozwala na tworzenie obszarów opływanych przez tekst
\usepackage{amsmath,amsfonts,amssymb} %rozszerza zbiór symboli matematycznych
\usepackage[rightcaption]{sidecap} %pozwala na tytuły obrazów - caption{}
%po prawej lub lewej stronie obrazu - wymaga opakowania w środowisko
\usepackage{float} % wymagany aby w niektórych środowiskach ustawiać opcje wydruku, np na górze strony, na dole, czy dokładnie tam gdzie został wstawiony w kodzie źródłowym
\usepackage[margin=0.5in]{geometry} %odpowiada za rozmiar marginesów
\usepackage{hyperref} %pozwala na tworzenie hiperłączy
\usepackage{xcolor}
\usepackage{fancyvrb}
\usepackage{tcolorbox}
\usepackage{titlesec}
\usepackage[acronym,toc,section=section]{glossaries}
% acronym - allows create different table for acronyms
% toc,section=section - allows glossaries to be printed in table of contents
\tcbuselibrary{skins,breakable} %tutaj nie pamiętam ale chyba te środowiska z czarnym tłem w których jest pisany kod
%==========================================================================
% DEFINICJE
% def
\def\bs{\textbackslash}
\def\us{\textunderscore}
\def\source_space{\vspace{0.2em}}

% definecolor
% href colors
\definecolor{hrefurl}{RGB}{46,71,217}
\definecolor{hreflink}{RGB}{39,117,15}

% newenvironment
\newenvironment{BGVerbatim} %to środowisko tworzy verbatim z czarnym tłem
 {\VerbatimEnvironment
  \begin{tcolorbox}[
    breakable,
    colback=vsblack,
    spartan
    ]%
  \begin{Verbatim}}
 {\end{Verbatim}\end{tcolorbox}}
 
% newcommand
\newcommand{\tc}[2]{\textcolor{#1}{#2}}
\newcommand{\ul}[1]{\underline{#1}}
\newcommand{\myline}[2]{\noindent\makebox[\linewidth]{\rule{#1cm}{#2pt}}}
\newcommand{\keyshortcut}[1]{\texttt{#1}}
%==========================================================================
% USTAWIENIA
\hypersetup %edytuje właściwości hiperłączy np. kolor
{
colorlinks=true,
linkcolor=black,
urlcolor=hrefurl
}

% \pagestyle{empty} %sprawia ze strony nie są numerowane - jezeli jest na stronie \maketitle to style jest nadpisywany i na takiej stronie trzeba dodatkowo umiescic zaraz po \maketitle komende \thispagestyle{empty}

\parindent 0px %ustawia wartość wcięcia na początku akapitu/paragrafu na zero, co daje taki efekt, że nie ma wcięć

\titleformat{\part}{\bfseries\centering\Huge{\titlerule[1.5pt]\vspace{0.1em}\titlerule[1.5pt]\vspace{0.5em}}}{Part \thepart\ -}{10pt}{\Huge}[{\vspace{0.5em}\titlerule[1.5pt]\vspace{0.1em}\titlerule[1.5pt]}]
\titlespacing{\part}{0pt}{0pt}{20pt}

\titleformat{\chapter}{\bfseries\centering\Huge{\titlerule[1.5pt]}}{\thechapter .}{10pt}{\huge}[{\titlerule[1.5pt]}]
\titlespacing{\chapter}{0pt}{0pt}{20pt}

\titleformat{\section}{\bfseries\large}{\thesection}{0.5em}{}[\titlerule]
\titlespacing{\section}{0pt}{1ex}{10pt} %how to use "plus" operator

%glossaries

\makeglossaries

\newglossaryentry{paradigm}
{
    name=paradigm,
    description={a typical example or pattern of something; a pattern or model}
}

\newacronym{cpu}{CPU}{Central Processing Unit}


\date{\vspace{-3em}}
\title{\vspace{-2em}{\Huge \texttt{>> Computer Science <<}}\vspace{-0.5em}}
\author{{\LARGE Created and typeset by: Tomasz Zdeb}}
%adding some negative vspace allows to controll the spaces in the title

% KONIEC PREAMBUŁY
%=======================================================================
\begin{document}
\maketitle
% \thispagestyle{empty} % \maketitle nadpisuje dzialanie \pagestyle{empty} dlatego musimy zaraz po tej komendzie skorzystać z ustawienia dla tej konkretnej strony

\tableofcontents
\newpage

\part{\LaTeX}
\section{Glossaries}

Glossaries are being defined as ``\textit{an alphabetical list of words related to a specific subject, text, or dialect, with explanations; a brief dictionary}". To automate glossaries generation and maintenence \LaTeX\ uses a package called: \textbf{glossaries}. Like in most cases, it is sometimes desired to specify additional parameters to the package, e.g. \textit{acronym}, \textit{toc}, \textit{section=section}. Which will be explained later in this section.\\

Before processing any code two things have to be mentioned. One: glossaries package require Perl interpreter to be present at the machine. Two: the package requires custom compilation scheme in order to take effect. Unfortunately there is no preset built into \textbf{TeXMaker} environment for a compilation scheme that will proces glossaries. A command chain has to be configured manually.

\begin{figure}[H]
\centering
\includegraphics[scale=0.6]{LaTeX/figures/user_command_glossaries_marked.png}
\caption{Location of \textbf{Edit User Commands} button in \textbf{TeXMaker} environment}
\end{figure}

In order to define the custom command pipeline, go to: \textbf{User} -> \textbf{User Commands} -> \textbf{Edit User Commands} and type below code into \textbf{command} field:
%\vspace{-0.5em} --------------------------------------------------------------------
\begin{verbatim}
pdflatex -synctex=1 -interaction=nonstopmode %.tex | makeglossaries % | pdflatex 
-synctex=1 -interaction=nonstopmode %.tex | "C:/Program Files (x86)/Adobe/Acrobat
Reader DC/Reader/AcroRd32.exe" %.pdf
\end{verbatim}

\begin{figure}[H]
\centering
\includegraphics[scale=0.6]{LaTeX/figures/custom_command_marked.png}
\caption{Layout of \textbf{Edit User Commands} window in \textbf{TeXMaker} environment}
\end{figure}

Pipe symbol is used to chain commands. Words preceded by pause: ``-'' are parameters passed to commands, which are words without any additions, placed at the beginning of every part - right after pipe or at the very beginning. Disassembly of this chain of commands allows to differentiate four differents parts here:
\begin{enumerate}
\item Generate \textbf{.pdf} file from the \textbf{.tex} file
\item Generate glossaries
\item Generate \textbf{.pdf} file from the \textbf{.tex} file
\item Display \textbf{.pdf}
\end{enumerate}

There are two types of entries that glossary package provides:
\begin{itemize}
\item glossaries
\item acronyms
\end{itemize}

Glossaries are being defined in the preamble after \texttt{\bs makeglossaries} command that has to be put before any of the entries. Syntax of an glossary and acronym entry is as shown below.

\begin{figure}[H]
\centering
\includegraphics[scale=1.0]{LaTeX/figures/glossary_definition.png}
\caption{Commands used to define glossaries and acronyms}
\label{fig:glossary_definition}
\end{figure}

To print a list of glossaries, \texttt{\bs printglossary} command is used. For an entry to be printed on the list, at least one reference to it in the text is needed. Otherwise even if defined in the preamble, the entry won't be printet on the list.

\begin{figure}[H]
\centering
\includegraphics[scale=1.0]{LaTeX/figures/printglossary.png}
\caption{Commands used to print glossary and acronym list}
\label{fig:printglossary}
\end{figure}

By default a list containing both glossaries and acronyms is printed. As shown later on figure \ref{fig:usepackage_glossaries} parameter \texttt{acronym} can be provided while invoking the package use, that allows to print separate list for acronyms and glossaries. To print the separate list for acronyms specify \texttt{type=\bs acronymtype} attribute in a separate \texttt{\bs printglossary} call as shown on figure \ref{fig:printglossary}, which also illustrate, how a custom title\footnote{Title displayed in the text} as well as token title\footnote{Title displayed in table of content} can be set. To allow glossary table to be included in table of contents an additional parameters: \texttt{toc} and \texttt{section=section} are needed when invoking \textbf{glossaries} package as shown on picture \ref{fig:usepackage_glossaries}.

\begin{figure}[H]
\centering
\includegraphics[scale=0.8]{LaTeX/figures/glossary_types.png}
\caption{List of acronyms and glossaries with use of \textbf{acronym} parameter for \textbf{glossaries} package, that allows for separate list for acronyms to be printed}
\end{figure}

To reference a glossary or acronym, several commands are used.

\begin{figure}[H]
\centering
\includegraphics[scale=1.0]{LaTeX/figures/reference_glossaries.png}
\caption{Commands used to reference glossaries and acronyms}
\end{figure}

Each entry can be referenced in several ways. Plural and singular forms as well as upper case and lowercase versions are avaliable for glossaries. Acronym references differ from each other by level of detail printed.

\begin{figure}[H]
\centering
\includegraphics[scale=1.0]{LaTeX/figures/glossary_calls.png}
\caption{Examples of glossaries and acronyms references}
\end{figure}

\begin{figure}[H]
\centering
\includegraphics[scale=1.0]{LaTeX/figures/usepackage_glossaries.png}
\caption{Command used to invoke the use of \textbf{glossaries} package with additional parameters passed to it}
\label{fig:usepackage_glossaries}
\end{figure}

There are many more options regarding to glossary package, i.e. commands used to customize display of the gloassaries, enable sorting, create custom glossaries, put alternative text in references to a glossary, etc. You can find them on	\href{https://www.ctan.org/pkg/glossaries}{CTAN}
\section{Quotes}

There are several types of quoting avaliable in \LaTeX.

Symbol that is placed on the same key as \textbf{tilde} (under or next to the \textbf{escape} button), is used to generate right facing quotes used in anglo-saxon quoting notation.

\fbox{\textcolor{red}{To finish}}

\section{Description list environment}

Next to \textbf{Itemize} and \textbf{Enumerate} there is \textbf{Description} list environment in \LaTeX . It's a glossary type of list environment consisting of key-value data records. e.g.

\begin{description}
\item[Item one] description of item one
\item[item two] description of item two
\item[item three] description of item three
\end{description} 

Above structure is generated by code:

\begin{verbatim}
\begin{description}
\item[Item one] description of item one
\item[item two] description of item two
\item[item three] description of item three
\end{description} 
\end{verbatim}

\section{Captions and labels}
\fbox{\textcolor{red}{remember to surround tables, figures etc. in their wrapper floatin environments like figure, table etc. and add the caption and label}}
\section{Don't break line here}
\fbox{\textcolor{red}{to instruct \LaTeX no to break line between some content use tilde, e.g. no\textasciitilde line\textasciitilde break}}
\section{How to insert tilde into output pdf doc}
\fbox{\textcolor{red}{To be implemented}}
\section{ref and hred referencing}
\fbox{\textcolor{red}{To be implemented}}
\section{MikTeX standard files}
\fbox{\textcolor{red}{To be implemented}}
\begin{verbatim}
C:\Program Files\MiKTeX\tex\latex\base
\end{verbatim}


 %LaTeX Part

%\part{Windows}

\subsubsection{Key Shortcuts}
\begin{itemize}
\item \keyshortcut{<Ctrl + R>} - starts \textbf{run window}
\item \keyshortcut{<Win + Shift + S>} runs \textbf{Snip \& Sketch} (screenshot)
\end{itemize}

\subsection{Run Commands}
\begin{itemize}
\item \keyshortcut{Run -> taskmgr} - runs \textbf{Task Manager}
\item \keyshortcut{Run -> winver} - runs simple program displaying \textbf{Windows} version on current machine
\item \keyshortcut{Run -> mspaint} - runs \textbf{Paint}
\item \keyshortcut{Run -> calc} - runs \textbf{Calculator}
\item \keyshortcut{Run -> control} - runs \textbf{Control Panel}
\end{itemize}

\subsection{Windows register}
\textcolor{red}{change the article - what registry is used for?}

It's a system database that contains most of systems configuration options. User or system settings as well as applications setting are stored in the register. Before \textbf{Windows 95} \texttt{.ini} configurations files were used (similarly to linux), but later all of the configuration data is stored in register. To modify the register a dedicated editor is used, passing \texttt{regedit} to \textbf{Windows Run} will launch it.\\

\fbox{ WARNING - Do not do things listed below! they're shown just for educational purposes}\\

\textbf{Adding control panel to context menu}
\begin{itemize}
\item find \texttt{HKEY{\us}CLASSES{\us}ROOT{\bs}Directory{\bs}Background{\bs}shell} key
\item create a key named \texttt{Control Panel} or \texttt{Run Control Panel} (this name will be displayed in the context menu)
\item create subkey named \texttt{command}
\item modify its value to \texttt{rundll32.exe shell32.dll,Control{\us}RunDLL}
\end{itemize}

\textbf{Changing default instalation path}

\begin{itemize}
\item find \texttt{HKEY{\us}LOCAL{\us}MACHINE{\bs}SOFTWARE{\bs}Microsoft{\bs}Windows{\bs}CurrentVersion} key
\item modify \texttt{ProgramFilesDir} and \texttt{ProgramFilesDir (x86)} to your desired new default installation paths
\end{itemize}

\fbox{WARNING} After changing default instalation path I had some issues with \textbf{.NET Core} apps development, they couldn't find the runtime which was installed in C:{\bs}Program Files. After moving \textbf{dotnet} directory to new path apps were running again, but \textbf{Visual Studio} had some issues like couldn't find project templates. For these reasons i reverted all the changes made in these keys.\\

\textbf{Setting custom logo in System Properties }
\begin{itemize}
\item find \texttt{HKEY{\us}LOCAL{\us}MACHINE{\us}SOFTWARE{\bs}Microsoft{\bs}Windows{\bs}CurrentVersion{\bs}OEMInformation} key
\item create new string value named \texttt{Logo}
\item set \texttt{Logo} value to path pointing to desired graphics with \texttt{.bmp} file extension
\end{itemize}

\textbf{Setting custom logscreen prompt}
\begin{itemize}
\item find \texttt{HKEY{\us}LOCAL{\us}MACHINE{\bs}SOFTWARE{\bs}Microsoft{\bs}Windows NT{\bs}CurrentVersion{\bs}Winlogon}
\item edit \texttt{LegalNoticeCaption} and \texttt{LegalNoticeText} to add title and text
\end{itemize}

\subsection{Winver}
passing winver to \textbf{Windows Run} will launch a program that displays information about \textbf{Windows} version on the machine. %Windows Part

%\input{Linux/Linux} %Linux Part

%\input{GeneralConcepts/GeneralConcepts} %General Concepts Part

\part{Additions} %Additions Part

%[resource]:           \url{https://youtu.be/TW_-ZSWKlTY}\\
%[resource]:           \url{https://www.jaknauczycsieprogramowania.pl/l}

\printglossary[title=Abbreviations used in the book, toctitle=Abbreviations,type=\acronymtype]

\printglossary[title=List of Terms, toctitle=Glossary] 

\end{document}