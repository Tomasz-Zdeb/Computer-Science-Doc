% POCZĄTEK PREAMBUŁY
\documentclass[11pt,a4paper]{article}
%======================================================================
%    Te trzy pakiety należy dołączyć aby wyświetlać polskie znaki
%----------------------------------------------------------------------
\usepackage[utf8]{inputenc} % `utf8` option to match Editor encoding
\usepackage[T1]{fontenc}
\usepackage{lmodern}
%======================================================================
% DOŁĄCZONE PAKIETY
\usepackage{graphicx} %pozwala na dodawanie obrazów
\usepackage{wrapfig} %pozwala na tworzenie obszarów opływanych przez tekst
\usepackage{amsmath,amsfonts,amssymb} %rozszerza zbiór symboli matematycznych
\usepackage[rightcaption]{sidecap} %pozwala na tytuły obrazów - caption{}
%po prawej lub lewej stronie obrazu - wymaga opakowania w środowisko
\usepackage{float} % wymagany aby w niektórych środowiskach ustawiać opcje wydruku, np na górze strony, na dole, czy dokładnie tam gdzie został wstawiony w kodzie źródłowym
\usepackage[margin=1in]{geometry} %odpowiada za rozmiar marginesów
\usepackage{hyperref} %pozwala na tworzenie hiperłączy
\usepackage{xcolor}
\usepackage{fancyvrb}
\usepackage{tcolorbox}
\tcbuselibrary{skins,breakable} %tutaj nie pamiętam ale chyba te środowiska z czarnym tłem w których jest pisany kod
%==========================================================================
% DEFINICJE
% def
\def\bs{\textbackslash}
\def\us{\textunderscore}
\def\source_space{\vspace{0.2em}}

% definecolor
% href colors
\definecolor{hrefurl}{RGB}{46,71,217}
\definecolor{hreflink}{RGB}{39,117,15}

% newenvironment
\newenvironment{BGVerbatim} %to środowisko tworzy verbatim z czarnym tłem
 {\VerbatimEnvironment
  \begin{tcolorbox}[
    breakable,
    colback=vsblack,
    spartan
    ]%
  \begin{Verbatim}}
 {\end{Verbatim}\end{tcolorbox}}
 
% newcommand
\newcommand{\tc}[2]{\textcolor{#1}{#2}}
\newcommand{\ul}[1]{\underline{#1}}
\newcommand{\myline}[2]{\noindent\makebox[\linewidth]{\rule{#1cm}{#2pt}}}
%==========================================================================
% USTAWIENIA
\hypersetup %edytuje właściwości hiperłączy np. kolor
{
colorlinks=true,
linkcolor=hreflink,
urlcolor=hrefurl
}

% \pagestyle{empty} %sprawia ze strony nie są numerowane - jezeli jest na stronie \maketitle to style jest nadpisywany i na takiej stronie trzeba dodatkowo umiescic zaraz po \maketitle komende \thispagestyle{empty}


\parindent 0px %ustawia wartość wcięcia na początku akapitu/paragrafu na zero, co daje taki efekt, że nie ma wcięć

\date{\vspace{-3em}}
\title{\vspace{-2em}\texttt{Computer Science}\\ \begin{large} Notes on different branches of computing\end{large}\vspace{-0.5em}}
\author{Created and typeset by: Tomasz Zdeb}
%adding some negative vspace allows to controll the spaces in the title

% KONIEC PREAMBUŁY
%=======================================================================
\begin{document}
\maketitle
% \thispagestyle{empty} % \maketitle nadpisuje dzialanie \pagestyle{empty} dlatego musimy zaraz po tej komendzie skorzystać z ustawienia dla tej konkretnej strony
\myline{16}{1}

\begin{scriptsize}
Source Materials:\\
\fbox{\url{https://youtu.be/TW_-ZSWKlTY}}\source_space\\
\fbox{\url{www.website.com}}
\end{scriptsize}

\myline{16}{1}

\section{Windows}
\subsubsection{Windows register}
It's a system database that contains most of systems configuration options. User or system settings as well as applications setting are stored in the register. Before \textbf{Windows 95} \texttt{.ini} configurations files were used (similarly to linux), but later all of the configuration data is stored in register. To modify the register a dedicated editor is used, passing \texttt{regedit} to \textbf{Windows Run} will launch it.\\

\textbf{Adding control panel to context menu}
\begin{itemize}
\item find \texttt{HKEY{\us}CLASSES{\us}ROOT{\bs}Directory{\bs}Background{\bs}shell} key
\item create a key named \texttt{Control Panel} or \texttt{Run Control Panel} (this name will be displayed in the context menu)
\item create subkey named \texttt{command}
\item modify its value to \texttt{rundll32.exe shell32.dll,Control{\us}RunDLL}
\end{itemize}

\textbf{Changing default instalation path}

\begin{itemize}
\item find \texttt{HKEY{\us}LOCAL{\us}MACHINE{\bs}SOFTWARE{\bs}Microsoft{\bs}Windows{\bs}CurrentVersion} key
\item modify \texttt{ProgramFilesDir} and \texttt{ProgramFilesDir (x86)} to your desired new default installation paths
\end{itemize}

\fbox{WARNING} After changing default instalation path I had some issues with \textbf{.NET Core} apps development, they couldn't find the runtime which was installed in C:{\bs}Program Files. After moving \textbf{dotnet} directory to new path apps were running again, but \textbf{Visual Studio} had some issues like couldn't find project templates. For these reasons i reverted all the changes made in these keys.\\

\textbf{Setting custom logo in System Properties }
\begin{itemize}
\item find \texttt{HKEY{\us}LOCAL{\us}MACHINE{\us}SOFTWARE{\bs}Microsoft{\bs}Windows{\bs}CurrentVersion{\bs}OEMInformation} key
\item create new string value named \texttt{Logo}
\item set \texttt{Logo} value to path pointing to desired graphics with \texttt{.bmp} file extension
\end{itemize}

\textbf{Setting custom logscreen prompt}
\begin{itemize}
\item find \texttt{HKEY{\us}LOCAL{\us}MACHINE{\bs}SOFTWARE{\bs}Microsoft{\bs}Windows NT{\bs}CurrentVersion{\bs}Winlogon}
\item edit \texttt{LegalNoticeCaption} and \texttt{LegalNoticeText} to add title and text
\end{itemize}

\subsubsection{Winver}
passing winver to \textbf{Windows Run} will launch a program that displays information about \textbf{Windows} version on the machine.

\section{Linux}

\section{GIT}

\end{document}