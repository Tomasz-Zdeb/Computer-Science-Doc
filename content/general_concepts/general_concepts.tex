\part{General Concepts}

\chapter{Learning Strategies}

\section{Top-down and bottom-up}

There are two main approaches when it comes to learning how to code:
\begin{itemize}
\item \textbf{top-down} - focuses on following a tutorial on how to make a full project without going too much into details 
\item \textbf{bottom-up} - focuses on learning basic concepts and all the details, then aggregating them into a bigger project
\end{itemize}
None of these approaches is perfect, both have their pros and cons, but to achieve quite a good learning effitiency it's optimal to combine these two in a learning process.\\

Building a working application gives a serious \textbf{load of satisfaction} that pushes the one deeper into learning process, that is the reason why learning should focus on \textbf{solving some real life problems}. It's good to learn programming in the incremental way - that means that it's necessary to always maintain working application, and develop only one (or even a part of) new feature at the time (to avoid scenarios when working on couple different features, and none of them is working as well as the whole application). This approach could be used not only in term of creating a single project but in term of whole learning process. Sometimes it's good to learn a little bit of one thing, then another, and then another, when gathered some general knowledge on these topics, come back to the first and master it, then to the second and then to the third.

\chapter{Content Versioning}

\section{Copying}

\section{Diff}

\section{GIT}

\section{SVN}

\chapter{What is...?}

\section{REST API}


Acronyms stand for \textbf{REpresentational State Transfer} and \textbf{Application Programming Interface}.\\

What's an API

The purpose of creating an \textbf{API} is to allow \textbf{application} or \textbf{service} access to a resource in other \textbf{application} or \textbf{service}. \textbf{Application} or \textbf{service }that contains the resource is then called \textbf{server}, and the \textbf{application} or \textbf{service} that accesses the resource is called \textbf{client}.\\

Stateles communication

\textbf{WEB APIs} don't have necessarily to be \textbf{RESTful}, the most characteristic feater of \textbf{REST} is that the communication is \textbf{stateless} (there is no session created, all the necessary information is contained in the passed request, and no information about previous requests is stored).\\

How todays REST APIs work

Most of todays \textbf{WEB Services} use \textbf{RESTful APIs} due to their flexibility and versatility. \textbf{REST} comparing to first generation \textbf{XML-RPC} protocol and second generation \textbf{SOAP} (that force to use a very specific structure of communication)	can be created by any programming language and use many different data formats, JSON is the most popular one though due to it's human readable form and simplicity.\\

\textbf{REST APIs} use \textbf{HTTP} requests to perform \textbf{CRUD} (\textbf{Create Read Update Delete}) opeartions on the resource. For example an \textbf{REST API} can use \texttt{GET} request to obtain a record, \texttt{POST} request to create a record, \texttt{PUT} request to update a record and \texttt{DELETE} request to delete a record.

Endpoints

To allow clients to acces the methods, a \textbf{path} is assigned to every method - this path is called the \textbf{endpoint}. Developers assign (map) these paths to a given method in \textbf{application} code.

\fbox{\textcolor{red}{Extract glossary entries}}\\

\section{SOAP}

\fbox{\textcolor{red}{populate with content}}\\

